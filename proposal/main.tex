\documentclass[11pt, oneside]{article}

% Required packages
\usepackage[letterpaper, margin=1.5in, includeheadfoot]{geometry}
\usepackage{hyperref}
\usepackage{tabularx}

% Header and Footer
\usepackage{fancyhdr}
\pagestyle{fancy}
\renewcommand{\headrulewidth}{1pt}
\lhead{}\chead{\textsc{CEE 588: Boundary Layer Meteorology $\cdot$ Princeton University}}\rhead{}
\lfoot{}\cfoot{\thepage}\rfoot{}

%\renewcommand{\baselinestretch}{1.25}

% Additional packages
\usepackage{graphicx}
\usepackage{amssymb}
\usepackage{amsmath}

\begin{document}

% Title and Author block
\begin{centering}
{\Large \textbf{Characterization and Forecasting of Wind Data}}\\
\vspace{\baselineskip}
{\large \textbf{Final Project Proposal}}\\
\vspace{\baselineskip}
Joseph G. Tylka\\
\href{mailto:josephgt@princeton.edu}{josephgt@princeton.edu}\\
\vspace{\baselineskip}
May 11\textsuperscript{th}, 2018\\
\end{centering}

%% Abstract
% The first sentence should be a very succinct statement of the problem; almost a rewording of the title with a few explanatory words
% The second sentence should give specific motivation for the work
% The following 2-3 sentences should be about the adopted method and approach
% The rest should be a statement of all the major findings
\begin{abstract}
Using experimentally measured wind data at various heights, parameters of the atmospheric boundary layer are estimated.
Existing methods for forecasting wind data rely on the roughness height and eddy viscosity.
In this work, these parameters are
Two well-established forecasting methods, persistence and autoregression, are implemented and evaluated.

\end{abstract}


\begin{enumerate}
\item Characterize and determine relevant timescales
\item Construct PCA-based model composed of principal components on each timescale
\item Alternatively, construct autoregressive model composed of AR coefficients on each timescale
\item Evaluate proposed model by comparison with real data
\end{enumerate}

According to the log-law, the wind magnitude is given by
\begin{equation}
|U(z)| = 
\end{equation}

Beginning with the expression for $\nu_T(z)$ given in Eq.~(21), we first differentiate $\nu_T(z)/z$ with respect to $z$, yielding
\begin{equation}
\frac{\partial}{\partial z} \left[ \frac{\nu_T(z)}{z} \right] = \overline{|A_g|} L_0 \frac{\partial}{\partial z} \left[ \frac{2}{z_i} - \frac{z}{z_i^2} \right] = -\frac{\overline{|A_g|} L_0}{z_i^2}.
\end{equation}
Substituting this into Eq.~(14) yields
\begin{equation}
\alpha(z) = \frac{\overline{|A_g|} L_0}{z_i^2 \log(z/z_0)}.
\end{equation}

\end{document}