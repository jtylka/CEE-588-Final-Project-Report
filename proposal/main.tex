\documentclass[11pt, oneside]{article}

% Required packages
\usepackage[letterpaper, margin=1.5in, includeheadfoot]{geometry}
\usepackage{hyperref}
\usepackage{tabularx}

% Header and Footer
\usepackage{fancyhdr}
\pagestyle{fancy}
\renewcommand{\headrulewidth}{1pt}
\lhead{}\chead{\textsc{CEE 588: Boundary Layer Meteorology $\cdot$ Princeton University}}\rhead{}
\lfoot{}\cfoot{\thepage}\rfoot{}

%\renewcommand{\baselinestretch}{1.25}

% Additional packages
\usepackage{graphicx}
\usepackage{amssymb}
\usepackage{amsmath}

\begin{document}

% Title and Author block
\begin{centering}
{\Large \textbf{Final Project Proposal}}\\
\vspace{\baselineskip}
Joseph G. Tylka\\
\href{mailto:josephgt@princeton.edu}{josephgt@princeton.edu}\\
\vspace{\baselineskip}
March 16\textsuperscript{th}, 2018\\
\end{centering}

%% Abstract
% The first sentence should be a very succinct statement of the problem; almost a rewording of the title with a few explanatory words
% The second sentence should give specific motivation for the work
% The following 2-3 sentences should be about the adopted method and approach
% The rest should be a statement of all the major findings
\begin{abstract}
A method for wind energy forecasting is proposed.
\end{abstract}

\end{document}